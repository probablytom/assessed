\documentclass[a4paper,portrait,12pt,draft]{article}
\usepackage[latin1]{inputenc}
\usepackage{calc}
\usepackage{setspace}
\usepackage{fixltx2e}
\usepackage{graphicx}
\usepackage{multicol}
\usepackage[round]{natbib}
\usepackage{csquotes}
\usepackage{color}
\usepackage{hyperref}
\usepackage[colorinlistoftodos,textsize=tiny,obeyDraft]{todonotes}
\usepackage{cleveref}
 
\author{William Wallis --- 2025138W}
\title{Reviewing the Systematic Review}

\setlength{\oddsidemargin}{0.9847in-1in}
\setlength{\textwidth}{\paperwidth{}-0.9847in-0.9847in}

\crefname{chapter}{\S}{\S\S}
\crefname{section}{\S}{\S\S}
\setcounter{secnumdepth}{3}
\crefname{table}{table}{tables}
\Crefname{table}{Table}{Tables}
\crefname{figure}{figure}{figures}
\Crefname{figure}{Figure}{Figures}

%% --------------------------------------------------------------------------------------------------------------------------------
\begin{document}
\title{Engineering Standards for Anthropomorphic Algorithms}
\author{Tom Wallis}
\date{}
\maketitle


%====================================================
\section{Proposed Approach}
\label{sec:proposed_approach}
%====================================================
\subsection{Abstract}\label{sec:abstract}
The field of human-like computing --- and the study of algorithms which mimic human behaviours especially\footnote{Henceforth referred to as ``Anthropomorphic Algorithms''.} --- is one of increasing importance in academic and industrial circles, and sees application in information security and human-computer interaction, as well as tangentially related fields, such as smart city development.\par

Developing these anthropomorphic algorithms can be complicated, however. They require rigorous study in social sciences such as psychology, anthropology, and sociology, as well as the understanding of artistic studies such as philosophy. This added complication means that pursuit of human-like computing requires interdisciplinary study to effectively research and implement these systems. This often results in simple models of one human behavioural trait, rather than more involved multiple-trait models.\par

% This limitation is problematic: the complexity of the field, and its obscurity relative to other fields such as pervasive computation or quantum computing paradigms, mean that many researchers are not drawn to the field as a possible opportunity.\par
% 
In this proposal, a research opportunity is presented which can solve both of these issues by producing currently absent tooling and methodologies for the field. Once successfully completed, the tools and methodologies produced would reduce the interdisciplinary complexity of the field, and create jargon and tools which reduce the friction involved in undertaking this research from multiple angles. These tools and methodologies would, in turn, permit currently difficult-to-pursue research engineering multiple-trait anthropomorphic algorithms. These models would strengthen the industrial utility of this field, as the utility of the models is compounded with more traits --- garnering more interest in the field, and putting existing research to better use, though applications in smart cities, voice assistants, and more.\par

\subsection{Problem Outline}\label{sec:problem_outline}
% a fair degree of research has been done in this field
% - lots in trust, other traits like comfort, reputation, regret

% this research is difficult
% - and while there's utility in the field, it's complicated and interdisciplinary

% It's made more difficult by the lack of 
% * a common jargon
% * a common way of solving problems
% * interdisciplinary-focused researchers who can understand multiple jargons and multiple problem-solving methods

% Unfortunately, it's not a simple case of just "making it simpler". 
% - we need to take the complexities of social science and humanities into account! We can't make humans and culture simpler. 
%   | ...so maybe we can simplify the engineering?
% - how do we simplify engineering, so we can focus on those complexities?

Human-like computing, as a research field, has grown significantly in recent years --- particularly with regards literature on trust formalisms. However, the field is unusual, in that the development of an anthropomorphic algorithm requires an understanding of not only computing science, but also social sciences: a formalism's accuracy depends on its psychological and sociological perspective. More anthropomorphic formalisms require an understanding of other fields, such as philosophy and ethnography also, as their modelling of human traits --- and affect on our culture --- is critical to understand during the model's creation. This interdisciplinary nature is one of the field's greatest strengths.\par

It is also one of its greatest weaknesses. The requirement for a formalism to have a well-defined psychological and sociological model, as well as potential ethnographic and philosophical perspectives, so as to be implemented and evaluated by a computing science researcher, means that only particularly polymathematical researchers can undertake the work --- assuming that it arouses their interest in the first place. The alternative, an interdisciplinary team who can perform the research with a shared understanding of different components of the formalism, has its own complications: communication can be hampered by the differences in different parties' jargons. Moreover, the aims of researchers with different backgrounds can differ: some fields, such as the social sciences, have an interest in modelling human activity accurately, but computing scientists and philosophers can find the models useful as a metaphor in their studies.\par

% Therefore, no suitable system currently exists for undertaking this research. Either a researcher adept in both computing science and social science is required, or a team with unusually good communication skills, each of the members of which should understand the (complicated) jargons of the others.\par

Moreover, this interdisciplinary disparity can create further tensions, as no guidelines exist on how these models should be implemented and tested. A team of differing backgrounds will naturally diverge on how a formalism should be evaluated, and its creation can be complicated by the lack of clear guidelines on its implementation and evaluation. A model of multiple human traits can quickly couple the many traits together, and should one trait need to be altered, this can ripple through a model to cause major setbacks. Given the difficulty communicating between team members, and the complexity of the project for a single researcher, these major setbacks should be expected at present.\par

Solving this problem has its own complexities. This particularly can be seen when analysing the intricate nature of psychological and sociological research on a single topic.\footnote{As an example, consider the differences between Luhmann's approaches to trust~\cite{luhmann2000familiarity} compared to Deutsch's~\cite{deutsch1962cooperation} Deutsch believed that trust was inherently a perspective of an individual regarding the world, whereas Luhmann's perspective centred around the broader-scale sociological impact that trust has.}. The problem can be tackled however --- as will be seen --- by separating the engineering from the theory, and creating guidelines and tooling which simplify the model's creation and structure.\par
\subsection{Approaching the Problem}\label{sec:approaching_the_problem}

% Our approach is to:
% - develop methodologies support anthropomorphic algorithm research and developing
% - develop tooling to support research and development according to the methodologies, and to simplify the overall structure of a model through this support
To approach the problem of simplifying the engineering, one must first analyse what trends exist currently --- a standard should fit the direction that the field is currently taking. Once undertaken, however, this analysis will lend itself to the creation of:
\begin{enumerate}
    \item A methodology for formalism creation and evaluation.\\
    This should help the engineering effort to get out of the way of the research, and will provide standards and guidelines for anthropomorphic algorithms research, as well as for the engineering afterwards.
    \item Tooling to support this methodology.\\
    This will help drive adoption of the methodology, as well as ensuring that evaluable, well-engineered formalisms are easier to create --- strengthening the case for industrial applications.
\end{enumerate}

\subsubsection*{Methodology/Guideline Component}\label{sec:methodology}
% Methodologies
% - identify similarities in:
%   | models' construction and the way they're used
%   | social science research used in model creation
% - Show how one framework can support lots of different models
% - create a jargon around that framework, and produce guidelines to simplify research and interdisciplinary communication
To create the methodology appropriate for solving the problem of the creation and engineering of these anthropomorphic algorithms, it will be important to analyse multiple aspects of the existing literature.\par

For example, it is critical that the methodologies and guidelines produced are in line with existing models. Moreover, it is vital that these methodologies and guidelines are suitable at a number of levels: they must support not only the engineering of a model, but the engineering of models with psychological, sociological, ethnographic or philosophical perspectives on their respective traits.\par

The methodologies and guidelines created would not only support the creation of a model, but would support the creation of several models, ideally of different traits. This would increase the degree to which different models can be compared, and can be used as the basis of other models.\par

Once these methodologies and guidelines exist, jargons around this framework can be made concrete, providing one jargon that all members of a research team investigating anthropomorphic algorithms and formalisms of human traits can learn. This would simplify the existing research, and solve some of the issues in interdisciplinary communication.\par

\subsubsection*{Tooling Component}\label{sec:tooling}
% Tooling
% - Once that methodology is created, create tools such as trait containers and design patterns to aid formalism production and design.
% - Produce example formalisms:
%   | single-trait
%   | multiple-trait
Once the methodology and guideline component of work is complete, tooling for the system can be constructed which supports this as a specification. This tooling would take the form of engineering techniques, such as appropriate design patterns, as well as libraries which encourage the construction and design of a formalism according to the guidelines.\par

Ideally, these guidelines would simplify model construction so as to greatly reduce the technological barrier to entry; perhaps enabling social sciences researchers to implement the models themselves without a dedicated software engineer/computing science researcher. Proper tooling enables these changes to occur, and will help to further ignite the field.\par

\subsection{Use Cases}
% passive authentication -- Tim's paper
\subsubsection*{Passive Authentication}
One possible application of this work would be simpler methods for using anthropomorphic algorithms in passive authentication. This would build on already published work applying anthropomorphic properties to device security\cite{Crawford2013127}, strengthening the system and allowing the formalisms used to be evaluated more reliably. Evaluation becomes more reliable as the engineering of the anthropomorphic algorithm becomes more reliable, and can be tested against other similar anthropomorphic algorithms engineered using the same techniques.\par

% Smart cities
\subsubsection*{Smart Cities / Urban Development}
Another potential application would be smart city development. Currently, many programmes exist for the development of smarter cities\cite{nam2011conceptualizing}, yet anthropomorphic algorithms see little application in the space of urban planning. Many applications of anthropomorphic algorithms can be envisaged\cite{wallis_x}, resilient city development and anthropomorphically responsible emergency response being two examples.\par

% Voice assistants
\subsubsection*{Voice Assistants}
Voice assistants in commercial electronics are becoming more popular, as can be seen from the rise in popularity of technologies such as Apple's ``Siri'', Amazon's ``Alexa'', and the Google Now assistant. From a human-computer interaction perspective, these technologies rely in part on a human-like interaction mechanism through the use of a human-like voice, and natural language parsing.\par

Designers can further the anthropomorphic metaphor used in many of these devices through the application of anthropomorphic algorithms --- however, no well-founded engineering standard exists for this technology, preventing its broader commercial application. A potential use case would be the integration of anthropomorphic algorithms to these consumer electronics products.\par

%====================================================
\section{Background}
\label{sec:background}
%====================================================
% There's a lot of variety in the related literature, because there's a variety of different traits that can be modelled. 
Related literature on anthropomorphic algorithms varies widely, due to the multitude of different traits which are formalised. However, the trait with the greatest degree of pre-existing literature is trust; this background survey will therefore cover the nuances of trust research particularly, but will also cover some of the other currently emerging trait models.\par

\subsection{Trust}\label{sec:trust}
% The biggest field of research is trust. 
% Lots of different approaches to trust modelling.
Many different approaches have been taken to trust modelling. For example, Marsh's seminal model of trust\cite{marsh1994} is largely founded in psychology and sociology research; Eigentrust\cite{Kamvar2003} instead uses trust modelling as a metaphor on which to base network security algorithms. This subsection will explore the myriad ways in which trust modelling can vary in its academic philosophy.\par

\subsubsection*{Marsh}\label{sec:marsh}
Marsh's model of trust is the first specifically computational perspective on a human behavioural trait\footnote{Birkhoff's early work on the mathematics of Aesthetics\cite{Birkhoff1933} being a possible exception, though Birkhoff's intention was not to create a computational model --- ``Aesthetic Measure'' pre-dates the Church-Turing thesis by about three years.}. It draws on psychological and sociological work to formalise a general theory of trust, and creates a mathematical representation of this theory. The theory is then tested by evaluating it in application to reinforcement learning agents.\par

One particularly interesting notion Marsh provides is that Trust can be considered from three degrees of detail:
\begin{itemize}
    \item Basic Trust\\
        This is an agent's general inclination to trust; from a human perspective, one might consider it the ``trusting-ness'' of the agent.
    \item General Trust\\
        This is an agent's inclination to trust another, specific agent. An example of general trust might be a student's degree of trust toward their thesis advisor --- the student might trust the advisor completely, not at all, or somewhere in-between.
    \item Specific Trust\\
        This is an agent's inclination to trust another, specific agent to enact a task or complete some goal. An example of Specific Trust might be that a student might trust their supervisor completely to write a reference, but not to fly a Boeing 777.
\end{itemize}

In creating these degrees of detail, Marsh attempts to replicate the way that human beings trust when considering different aspects of a scenario. One can imagine asserting, ``He's rather trusting'', ``She trusts her advisor'', or ``they don't trust each other to write a grant proposal'' --- each having its own related, yet distinct meaning. In integrating this notion into his model of trust, Marsh creates an algorithm which satisfies the literature he cites for both psychological and sociological perspectives on trust.\par

Worth noting is that Marsh's model permits graded degrees of trust for all of these types. That is to say, Marsh's model can associate numbers to its measurement of trust, and so weigh up different degrees of trust to make decisions.\par

\subsubsection*{C\&F}\label{sec:cnf}
Castelfranchi and Falcone\cite{Castelfranchi2001} --- abbreviated to C\&F in popular literature --- created their own model of trust, based on a logical formalism of social trust. Their formalism defines logical predicates which define the nature of trust in terms of confidence, dependence, and disposition, which are also defined by them in logical terms.\par

The degree of detail Castelfranchi and Falcone provide creates a simple method for calculating and assessing whether one agent trusts another, in a boolean fashion, and in a way which scales to complicated multi-agent systems (or ``MAS'') effectively. This is more difficult in Marsh's model of trust, which is designed to measure degrees of trust from multiple agents' perspectives, and involves more complex calculation as a result.\par

C\&F, however, has its own problems. For example, the boolean logical approach that C\&F provide makes gradations of trust difficult to calculate. While they provide one possible approach at the end of their original paper, the elegance and simplicity of the non-graded approach is lost to the additional detail.\par

C\&F in its non-graded form therefore has middling applicability in the real world; however, it has served as the basis of much further work\cite{Herzig2009}, including more powerful modal logics.\cite{kramdi}\par

It is clear to see that, between even these two early models of only one trait, a significant degree of difference exists in the implementation detail of the two formalisms. Moreover, their philosophies with regards social sciences differ; Marsh taking an approach between sociology and psychology, C\&F leaning more to the sociological aspects. However, both models are created with the intention of directing the behaviour of intelligent agents. Therefore, at least some similarities exist which might be used to construct methodologies and guidelines around.\par

\subsubsection*{Eigentrust}\label{sec:eigentrust}
In comparison to the social sciences-oriented approaches taken by Marsh and C\&F, Eigentrust\cite{Kamvar2003} takes a very different approach. Eigentrust leverages a reputation-centric approach, basing its algorithms on star ratings often used by e-commerce platforms such as eBay and Amazon for rating their traders. The core of the algorithm rests on a very simple calculation of overall satisfaction:
\[s_{i, j} = sat(i, j) - unsat(i, j)\]
\ldots{}where \(sat\) is a function representing the number of interactions agent \(i\) has had with agent \(j\) which they deem ``satisfactory'', and \(unsat\) the function representing the number of interactions deemed ``unsatisfactory''. From this simple formula, Eigentrust calculates agent \(j\)'s reputation from the perspective of agent \(i\), and through a series of more complex transformations using linear algebra, arrives at a distributed ledger of reputation scores which takes into account the input from all agents.\par

Eigentrust is therefore different to Marsh and C\&F's respective formalisms in important ways. Examples include:
\begin{itemize}
    \item Networking agents together\\
        Agents under Eigentrust discuss each others' assessments of reputation with each other; to facilitate this, protocols for communicating reputation scores between these agents must be established. Unlike models created by Marsh and C\&F, Eigentrust permits an agent to trust or distrust another agent by taking into account the interactions the other agent has had with third parties.
    \item Evaluation\\
        The intended application of Eigentrust is directing the behaviour of an intelligent agent, but unlike Marsh or C\&F's models, Eigentrust is designed to be directly applied in areas such as network security. While C\&F and Marsh can evaluate their models based on whether the actors behave in more ``trusting'' ways, for some metric of trust, Eigentrust is evaluated via its efficacy in protecting nodes on a network from unsatisfactory interactions (such as downloading viruses or receiving bad packets).
    \item Trust as a useful metaphor, rather than creating a realistic anthropomorphic model\\
        One can see from this difference in evaluation that Eigentrust is fundamentally an application of trust as a useful metaphor in security engineering. Eigentrust's goal is not a realistic representation of trust, but a more limited application of trust to fulfil a specific purpose. 
\end{itemize}

These differences would, at first glance, indicate that differences between formalisms can be wide-reaching enough that one set of methodologies or guidelines for their research would not cover models for even one trait; however, while their implementation and application \emph{do} vary considerably, Eigentrust's direction of agent behaviour indicates that some middle ground can indeed be reached. However, due to the broad spectrum of possible formalisms a set of methodologies and guidelines would need to cover, one must be particularly familiar with a range of formalisms to successfully cover this range. The work to be undertaken, therefore, is far from trivial, and would require a great degree of time to cover properly.\par

\subsection{Reputation}
Rather than using reputation to bootstrap a model of trust, some formalisms simply model reputation. REGRET\cite{Sabater} from Sabater \& Sierra is one such formalism.\par

The perspective of REGRET is that reputation is the ``opinion or view of one (agent) about something''\cite{Sabater}. This definition of reputation has no specific roots in psychology or sociology; however, REGRET makes use of social sciences research indicating that social structure and recency of events are important factors in an anthropomorphic agent's assessment of reputation. The authors' philosophy is that reputation is fundamentally an opinion; therefore, it is unsurprising that a main focus of this model is the successful modelling of an agent's opinion. Other factors assessed include ``social reputation'', being an agent's inclination to inherit opinions which other agents hold (where some social connection between the two exists).\par

REGRET lies in the middle of the spectrum carved by socially anthropomorphic models, such as Marsh's, and metaphorically anthropomorphic models, such as Eigentrust. Due to its practical nature, it is easy to evaluate and to implement in real-world use cases; however, it draws on social sciences research heavily enough to have a particular lean toward a socially centred model. This balance makes REGRET a good example of a middle ground which can be carved between models.\par

REGRET also presents an important use case of the guidelines and methodologies produced as a part of the proposed research. A model such as Eigentrust, in its original detail, is tightly coupled to its implicit model of reputation; future work might combine the reputation modelling provided by REGRET into Eigentrust's calculation of graded trust, so as to explore how Eigentrust fares when modelling responsibility differently. Current approaches to the problem of anthropomorphic algorithm design and engineering do not present elegant methods for combining the different aspects of Eigentrust in a modular, uncoupled way. However, with appropriate guidelines, methodologies and tooling, this limitation of the current status quo need not create more difficult future research; however, these methods produced must apply not only to Trust modelling, but to Reputation, a largely unrelated behavioural trait, also.\par

\subsection{Responsibility}
Similarly to Trust modelling, some varied work has been done in the pursuit of responsibility formalisms.

\subsubsection*{Deontic Logic}

An early mathematical framework of responsibility can be found in the popular deontic logic\cite{deontic-logic}. Deontic Logic is a modal logic, similar to Kramdi's modal logic for trust.\cite{kramdi} It is worth noting that Kramdi's logic for trust is substantially more advanced than deontic logic is for responsibility --- therefore, while deontic logic is elegant in its simplicity, it is of limited practical use.\par

Some attempts are made to overcome this. DeLima et al., for example, create more complex responsibility models which are able to model the allocation and discharge of tasks with greater detail, even in complex MAS.\cite{DeLima2008} However, further research is yet required to develop DeLima et al.~'s model into a fully featured anthropomorphic algorithm for responsibility.

\subsubsection*{Wallis}
Lately, some attention has been paid to responsibility modelling via anthropomorphic algorithms.\cite{wallis2017} While previously logical models of responsibility has been developed~\cite{berreby2015modelling}, responsibility formalism work currently underway (by the researchers involved in this grant proposal) provide a socially inspired responsibility formalism in the same vein as Marsh's trust modelling formalism.\par

This work presents a model of responsibility inspired heavily by sociotechnical systems research --- particularly that of Ian Sommerville\cite{Sommerville:2007ec}. In sociotechnical literature, responsibilities are actions which can be discharged; the history of discharged responsibilities discussed as ``consequential'' responsibilities, and obligation to act in the future identified as ``causal'' responsibilities. Similar demarcations are made in philosophical literature by Scanlon\cite{scanlon2006justice} and by Strawson\cite{strawson}.\par

As indicated earlier --- and as a result of its philosophical inspiration --- Wallis' model of responsibility lies on the Marsh and C\&F side of the aforementioned spectrum. This might be to be expected: where C\&F might consider trust as the belief that another agent will act, Wallis' model of responsibility treats a responsibility as an obligation to act in a certain way. The two traits can, by certain philosophies, be seen as very similar: Wallis' model of responsibility might work particularly well with Marsh's model of trust, for example. By contrast, there is very little similarity in Eigentrust's model of trust and Wallis' model of responsibility. Therefore, the approach one takes when initially creating a model is important; should the software engineer in a research team be unable to grasp the nuances of the social science aspect of a formalism, it may fail to fulfil its requirements simply by being insufficiently similar to the perspectives of other formalisms it is designed to work alongside.\par

\subsection{Discussion}
As can be seen, there exists a great breadth in the nature and detail of different anthropomorphic algorithms.\par

Some models use their respective traits as a metaphor by which they can represent some human-like behaviour; other models use research in philosophy and social sciences for an anthropomorphic realism. Even within the latter camp, differences in perspective with regards social science research can greatly affect the nature of the resulting model.\par

The creation of a set of guidelines by which this broad research can be carried out is clearly a complex task, which requires a great deal of interdisciplinary knowledge, and a particularly communicative and open-minded set of researchers to enact. However, the research is also urgent: in only 23 years since Marsh's seminal computational model of trust, the field has exploded into myriad forms, which vary wildly. To combat further divergence of the field, which would require even more work to generalise into methodologies and guidelines, this research should be undertaken as soon as possible.\par

Fortunately, some details of these models are constant. For example, all of the models presented are used in the direction of intelligent agent behaviour.\footnote{This is to be expected, given that they are formalisms of human behaviour, and are therefore primarily useful to direct the behaviour of other agents.} This similarity implies that a common thread between different models can be leveraged for the creation of guidelines and methodologies for anthropomorphic algorithm research and creation.\par

\section{Methodology}
\label{sec:methodology}

\subsection{Extensive Background Survey}
To create a set of guidelines and methodologies which appropriately support the creation of a range of models, and encourage similar architectures and formats for these models, existing literature will be extensively covered and analysed. Existing models will be evaluated independently, allowing the researchers involved to better understand the natures of the different models. No such complete review of the anthropomorphic algorithm field currently exists --- however, this well-founded basis for the research is critical to its successful completion.\par

After this unit of work, implementations of various models will be produced, and these models classified according to various useful metrics and factors so as to aid the guideline and methodology creation.

\subsection{Guideline and Methodology Creation}
Once a particularly detailed classification of different models is created, this will be used for the identification of useful guidelines and methodologies which can be used in the research and design of anthropomorphic algorithms.\par

This unit of work will produce design patterns, research methodologies, and other useful technologies, summarised and presented in a series of conference presentations and resulting papers.\par

\subsection{Tool Creation}
As the guidelines and methodologies which are created are finished, a library of tools which aid the creation of anthropomorphic algorithms according to these methodologies and guidelines will be constructed. This unit of work will produce these libraries.\par

\subsection{Tool Usage Examples}
Once the libraries are created, previous implementations of existing models will be re-engineered to make use of the new technologies. Further examples will then be developed, such as smart city development, voice assistant integration, and passive authentication.\par

This unit of work will present these examples, with the existing models showing the academic impact of the work produced, and the integration with commercial interests showing the industrial impact of the work. Consumer electronics companies and cities interested in smart technologies will be offered the opportunity to collaborate on these applications, enabled by the EPSRC grant.\par

% How are we going to put the tools together?

% Identify commonalities between the models and the traits
% - Analysis of the models
% - Analysis of the traits being modelled

% Produce guidelines for model creation based on how thee models are made / have been made / can & should be made

% Produce design patterns for research and models based on these models

% Produce tooling and libraries which support these design patterns

% Produce source code for existing models created against these guidelines using these tools

% Produce new models which are suitable for industry-setting deployment
% - either smart cities or voice assistants

\section{Risks}
\label{sec:risks}
% The work's complex -- one set of guidelines might not be enough

% Creating these tools doesn't mean people will adopt them -- we'd need to get the word out, and that can cost! We'd need money for conferences and things. 

As can be expected, this research is not without a degree of risk. One such risk is inherent in the interdisciplinary nature of the work: a detailed enough understanding of the field may be difficult to ascertain, though the history of the researchers involved is particularly strong.\par

Another risk worth noting is that the creation of these tools does not imply adoption. One use of the grant funds, therefore, will be the presentation of these tools, guidelines and methodologies in influential conferences, so as to encourage adoption and set a standard for the field as it progresses.\par

\section{Impact}
\label{sec:proposed_approach}
% Impact in two ways!
This research, due to its academic nature and potential to disrupt commercial spaces, can affect both academia and industry; it also carries national importance, as can be seen in relation to EPSRC's current interests.

\subsection{National Importance}
\label{sec:national_importance}
% Relates to EPSRC's push in ICT themes
% Improvements to British technological advancements
% Improvements in British urban development, and consumer electronics
EPSRC's theme of Information and Communication Technologies clearly relate to the proposed work, in two important respects:
\begin{itemize}
    \item The Informatics research involved in the proposed work is undoubtedly groundbreaking, and pushes the state of the art forward while guiding the next generation of research in the area
    \item Anthropomorphic algorithms have a great utility in communications technology, as they allow a device to communicate with a user in unusually human-centric ways.\cite{Crawford2013127}
\end{itemize}

These advancements notwithstanding, Britain has a long history of pursuing the very forefront of emerging computing technologies; anthropomorphic algorithms clearly represent the next stage in this tradition, and Britain can currently pave a way as a global leader in this emerging area of research.

\subsubsection*{Commercial Potential}
An aspect of this national importance is the commercial potential which the research represents. As Britain would be carving the path for both research and industry with the proposed work, an opportunity presents itself to lead the pack in commercial applications of these technologies.\par

Examples of the commercial viability of the work are provided as part of the delivered research; some of the funds provided by the EPSRC grant will be used for the integration of these technologies with consumer electronics companies and urban planning departments in Britain, so as to pursue some of the use cases outlined in this proposal.\par

\subsection{Academic Impact}
\label{sec:academic_impact}
% Ties well into EPSRC's intention to grow funding opportunities in software engineering!
% Improvements to CS research, and HCI, and software engineering
% Improvements to many other fields, such as psych, socio, geography, philosophy, ethnography...
EPSRC's current intention to grow its investment in software engineering research relates particularly well with this proposal. Not only would this research explore software engineering in a field previously neglected by engineering research, but this engineering has the opportunity to impact multiple aspects of computing science: improving the standards of not only software engineering, but enabling better research in fields such as human-computer interaction also. The return on investment of this particular body of work can therefore be expected to be unusually high --- especially when accounting for the commercial impact of the work, too.\par

This research does not only impact a range of computing science areas, however. Due to its interdisciplinary nature, better guidelines and methodologies for anthropomorphic algorithm research --- and tooling to make model creation vastly simpler --- allows researchers in other fields the opportunity to benefit from this work also. One can expect impacts from this research in philosophy of mind and political philosophy, as well as conflict resolution and trait modelling in psychology and sociology, urban planning opportunities in geography, and a greater toolset by which to analyse culture development in ethnography. 

\let\oldbibliography\thebibliography
\renewcommand{\thebibliography}[1]{\oldbibliography{#1}
\setlength{\itemsep}{-3pt}}

\bibliographystyle{abbrv}
%\setstretch{0.8}
{
\scriptsize
\bibliography{biblio}
}
\end{document}
