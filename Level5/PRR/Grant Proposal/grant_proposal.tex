\documentclass[a4paper,portrait,12pt,draft]{article}
\usepackage[latin1]{inputenc}
\usepackage{calc}
\usepackage{setspace}
\usepackage{fixltx2e}
\usepackage{graphicx}
\usepackage{multicol}
\usepackage[round]{natbib}
\usepackage{csquotes}
\usepackage{color}
\usepackage{hyperref}
\usepackage[colorinlistoftodos,textsize=tiny,obeyDraft]{todonotes}
\usepackage{cleveref}
 
\author{William Wallis --- 2025138W}
\title{Reviewing the Systematic Review}

\setlength{\oddsidemargin}{0.9847in-1in}
\setlength{\textwidth}{\paperwidth{}-0.9847in-0.9847in}

\crefname{chapter}{\S}{\S\S}
\crefname{section}{\S}{\S\S}
\setcounter{secnumdepth}{3}
\crefname{table}{table}{tables}
\Crefname{table}{Table}{Tables}
\crefname{figure}{figure}{figures}
\Crefname{figure}{Figure}{Figures}

%% --------------------------------------------------------------------------------------------------------------------------------
\begin{document}
\title{Engineering Standards for Anthropomorphic Algorithms}
\author{Tom Wallis}
\date{}
\maketitle


%====================================================
\section{Proposed Approach}
\label{sec:proposed_approach}
%====================================================
\subsection{Abstract}\label{sec:abstract}
The\todo{try to keep this section to one page!} field of human-like computing --- and the study of algorithms which mimic human behaviours especially\footnote{Henceforth referred to as ``Anthropomorphic Algorithms''.} --- is one of increasing importance in academic and industrial circles. Academic circles are increasingly looking to use the metaphor of anthropomorphic behaviour in information security, human-computer interaction, and fields tangentially related to computing science such as urban planning and smart city development.\par

Developing these anthropomorphic algorithms can be complicated, however. They require rigorous study in social sciences such as psychology, anthropology, and sociology, as well as the understanding of artistic studies such as philosophy. This added complication means that pursuit of human-like computing requires interdisciplinary study to effectively research and implement these systems. This complexity often results in simple models of one human behavioural trait, rather than more involved multiple-trait models.\par

This limitation is problematic: the complexity of the field, and its obscurity relative to other fields such as pervasive computation or quantum computing paradigms, mean that many researchers are not drawn to the field as a possible opportunity.\par

In this proposal, a research opportunity is described which can solve both of these issues by producing currently absent tooling and methodologies for the field. Once successfully completed, the tools and methodologies produced would reduce the interdisciplinary complexity of the field, and create jargon and tools which reduce the friction involved in undertaking this research from multiple angles. These tools and methodologies would, in turn, permit currently difficult-to-pursue research which engineers multiple-trait anthropomorphic algorithms. These models would strengthen the industrial utility of this field, as the utility of the models is compounded with more traits --- garnering more interest in the field, and putting existing research to better use, though applications in smart cities, voice assistants, and more.\par

\subsection{Problem Outline}\label{sec:problem_outline}
% a fair degree of research has been done in this field
% - lots in trust, other traits like comfort, reputation, regret

% this research is difficult
% - and while there's utility in the field, it's complicated and interdisciplinary

% It's made more difficult by the lack of 
% * a common jargon
% * a common way of solving problems
% * interdisciplinary-focused researchers who can understand multiple jargons and multiple problem-solving methods

% Unfortunately, it's not a simple case of just "making it simpler". 
% - we need to take the complexities of social science and humanities into account! We can't make humans and culture simpler. 
%   | ...so maybe we can simplify the engineering?
% - how do we simplify engineering, so we can focus on those complexities?

Human-like computing, as a research field, has grown significantly in recent years --- particularly with regards literature on trust formalisms. However, the field is unusual, in that the development of an anthropomorphic algorithm requires an understanding of not only computing science, but also social sciences: a formalism's accuracy depends on its psychological and sociological perspective. More anthropomorphic formalisms require an understanding of other fields, such as philosophy and ethnography also, as their modelling of human traits --- and affect on our culture --- is critical to understand during the model's creation. This interdisciplinary nature is one of the field's greatest strengths and most curious aspects.\par

It is also one of its greatest weaknesses. The requirement for a formalism to have a well-defined psychological and sociological model, as well as potential ethnographic and philosophical perspectives, so as to be implemented and evaluated by a computing science researcher, means that only particularly polymathematical researchers can undertake the research --- assuming that it arouses their interest in the first place. The alternative, an interdisciplinary team who can perform the research with a shared understanding of different components of the formalism, has its own complications, communication can be hampered by the differences in different parties' jargons. Moreover, the aims of researchers with different backgrounds can differ: some fields, such as the social sciences, have an interest in modelling human activity accurately, but computing scientists and philosophers can find the models useful as a metaphor in their studies --- as a thought experiment for philosophers, and in human-computer interaction for computing scientists.\par

Therefore, no suitable system currently exists for undertaking this research. Either a researcher adept in both computing science and social science is required, or a team with unusually good communication skills, each of the members of which should understand the (complicated) jargons of the others.\par

Moreover, this interdisciplinary disparity can create further tensions, as no guidelines exist on how these models should be implemented and tested. A team of differing backgrounds will naturally diverge on how a formalism should be evaluated, and its creation can be complicated by the lack of clear guidelines on its implementation and evaluation. A model of multiple human traits can quickly couple the many traits together, and should one trait need to be altered, this can ripple through a particularly sophisticated model to cause major setbacks. Given the difficulty communicating between team members, and the complexity of the project for a single researcher, these major setbacks should be expected at present.\par

Solving this problem has its own complexities. This particularly can be seen when analysing the intricate nature of psychological and sociological research on a single topic.\footnote{As an example, consider the differences between Luhmann's approaches to trust~\cite{luhmann2000familiarity} compared to Deutsch's~\cite{deutsch1962cooperation} Deutsch believed that trust was inherently a perspective of an individual regarding the world, whereas Luhmann's perspective centred around the broader-scale sociological impact that trust has.}. The problem can be tackled however --- as will be seen --- by separating the engineering from the theory, and creating guidelines and tooling which simplify the model's creation and structure.\par
\subsection{Approaching the Problem}\label{sec:approaching_the_problem}

% Our approach is to:
% - develop methodologies support anthropomorphic algorithm research and developing
% - develop tooling to support research and development according to the methodologies, and to simplify the overall structure of a model through this support
To approach the problem of simplifying the engineering, one must first analyse what trends exist currently --- a standard should fit the direction that the field is currently taking. Once undertaken, however, this analysis will lend itself to the creation of:
\begin{enumerate}
    \item A methodology for formalism creation and evaluation.\\
    This should help the engineering effort to get out of the way of the research, being the definition and evaluation of a formalism of a trait.
    \item Tooling to support this methodology.\\
    This will help drive adoption of the methodology, as well as ensuring that evaluable, well-engineered formalisms are easier to create --- strengthening the case for industrial applications.
\end{enumerate}

\subsubsection*{Methodology/Guideline Component}\label{sec:methodology}
% Methodologies
% - identify similarities in:
%   | models' construction and the way they're used
%   | social science research used in model creation
% - Show how one framework can support lots of different models
% - create a jargon around that framework, and produce guidelines to simplify research and interdisciplinary communication
To create the methodology appropriate for solving the problem of the creation and engineering of these anthropomorphic algorithms, it will be important to analyse multiple aspects of the existing literature.\par

For example, it is critical that the methodologies and guidelines produced are in line with existing models. Moreover, it is vital that these methodologies and guidelines are suitable at a number of levels: they must support not only the engineering of a model, but the engineering of models with psychological, sociological, ethnographic or philosophical perspectives on their respective traits.\par

The methodologies and guidelines created would note only support the creation of a model, but would support the creation of several models, ideally of different traits. This would increase the degree to which different models can be compared, and can be used as the basis of other models.\par

Once these methodologies and guidelines exist, jargons around this framework can be made concrete, providing one jargon that all members of a research team investigating anthropomorphic algorithms and formalisms of human traits can learn. This would simplify the existing research, and solve some of the issues in interdisciplinary communication.\par

\subsubsection*{Tooling Component}\label{sec:tooling}
% Tooling
% - Once that methodology is created, create tools such as trait containers and design patterns to aid formalism production and design.
% - Produce example formalisms:
%   | single-trait
%   | multiple-trait
Once the methodology and guideline component of work is complete, tooling for the system can be constructed which supports this as a specification. This tooling would take the form of engineering techniques, such as appropriate design patterns, as well as libraries which encourage the construction and design of a formalism according to the guidelines.\par

Ideally, these guidelines would simplify model construction so as to greatly reduce the technological barrier to entry; perhaps enabling social sciences researchers to implement the models themselves without a dedicated software engineer/computing science researcher. The feasibility of this goal is dependant on the results of an in-depth background survey, which would identify the complexity of the methodologies and guidelines, as well as whether one set of methodologies and guidelines can sufficiently cover all anticipated models.\par

An aspect of this tooling component would be demonstration that the tools would work; this can be done in two stages. 
\begin{enumerate}
    \item Re-implementation/redesign of existing models according to the methodologies and guidelines, built using the tools created.
    \item An industrial proof-of-concept, using the tools to create models with the simplicity, reliability, and power to have commercial application.
\end{enumerate}

\subsection{Use Cases}
% passive authentication -- Tim's paper
One possible application of this work would be simpler methods for using anthropomorphic algorithms in passive authentication. This would build on already published work applying anthropomorphic properties to device security\cite{Crawford2013127}, strengthening the system and allowing the formalisms used to be evaluated more reliably. Evaluation becomes more reliable as the engineering of the anthropomorphic algorithm becomes more reliable, and can be tested against other similar anthropomorphic algorithms engineered using the same techniques.\par

% Smart cities
Another potential application would be smart city development. Currently, many programmes exist for the development of smarter cities\cite{nam2011conceptualizing}, yet anthropomorphic algorithms see little application in the space of urban planning. Many applications of anthropomorphic algorithms can be envisaged\cite{wallis_x}, resilient city development and anthropomorphically responsible emergency response being two examples.\par

% Voice assistants
Voice assistants in commercial electronics are becoming more popular, as can be seen from the rise in popularity of technologies such as Apple's ``Siri'', Amazon's ``Alexa'', and the Google Now assistant. From a human-computer interaction perspective, these technologies rely in part on a human-like interaction mechanism through the use of a human-like voice, and natural language parsing.\par

Designers can further the anthropomorphic metaphor used in many of these devices through the application of anthropomorphic algorithms --- however, no well-founded engineering standard exists for this technology, preventing its broader commercial application. A potential use case would be the integration of anthropomorphic algorithms to these consumer electronics products.\par

%====================================================
\section{Background}
\label{sec:background}
%====================================================
% There's a lot of variety in the related literature, because there's a variety of different traits that can be modelled. 
Related literature on anthropomorphic algorithms varies widely, due to the multitude of different traits which are formalised. However, the trait with the greatest degree of pre-existing literature is trust; this background survey will therefore cover the nuances of trust research particularly, as an example of the different natures of anthropomorphic algorithms which the methodologies and tools would need to support.\par

\subsection*{Trust}\label{sec:trust}
% The biggest field of research is trust. 
% Lots of different approaches to trust modelling.
Many different approaches have been taken to trust modelling. For example, Marsh's seminal model of trust\cite{marsh1994} is largely founded in psychology and sociology research; Eigentrust\cite{Kamvar2003} instead uses trust modelling as a metaphor on which to base network security algorithms. This subsection will explore the myriad ways in which trust modelling can vary in its academic philosophy.\par

\subsubsection*{Marsh}\label{sec:marsh}
Marsh's model of trust is the first specifically computational perspective on a human behavioural trait\footnote{Birkhoff's early work on the mathematics of Aesthetics\cite{Birkhoff1933} being a possible exception, though Birkhoff's intention was not to create a computational model --- ``Aesthetic Measure'' pre-dates the Church-Turing thesis by about three years.}. It draws on psychological and sociological work to formalise a general theory of trust, and creates a mathematical representation of this theory. The theory is then tested by evaluating it in application to reinforcement learning agents.\par

One particularly interesting notion Marsh provides is that Trust can be considered from three degrees of detail:
\begin{itemize}
    \item Basic Trust\\
        This is an agent's general inclination to trust; from a human perspective, one might consider it the ``trusting-ness'' of the agent.
    \item General Trust\\
        This is an agent's inclination to trust another, specific agent. An example of general trust might be a student's degree of trust toward their thesis advisor --- the student might trust the advisor completely, not at all, or somewhere in-between.
    \item Specific Trust\\
        This is an agent's inclination to trust another, specific agent to enact a task or complete some goal. An example of Specific Trust might be that a student might trust their supervisor completely to write a reference, but not to fly a Boeing 777.
\end{itemize}

In creating these degrees of detail, Marsh attempts to replicate the way that human beings trust when considering different aspects of a scenario. One can imagine asserting, ``He's rather trusting'', ``She trusts her advisor'', or ``they don't trust each other to write a grant proposal'' --- each having its own related, yet distinct meaning. In integrating this notion into his model of trust, Marsh creates an algorithm which satisfies the literature he cites for both psychological and sociological perspectives on trust.\par

Worth noting is that Marsh's model permits graded degrees of trust for all of these types. That is to say, Marsh's model can associate numbers to its measurement of trust, and so weigh up different degrees of trust to make decisions.\par

\subsubsection*{C\&F}\label{sec:cnf}
Castelfranchi and Falcone\cite{Castelfranchi2001} --- abbreviated to C\&F in popular literature --- created their own model of trust, based on a logical formalism of social trust. Their formalism defines logical predicates which define the nature of trust in terms of confidence, dependence, and disposition, which are also defined by them in logical terms.\par

The degree of detail Castelfranchi and Falcone provide creates a simple method for calculating and assessing whether one agent trusts another, in a boolean fashion, in a way which scales to complicated multi-agent systems (or ``MAS'') effectively. This is more difficult in Marsh's model of trust, which is designed to measure degrees of trust from multiple agents' perspectives, and involves more complex calculation as a result.\par

C\&F, however, has its own problems. For example, the boolean logical approach that C\&F provide makes gradations of trust difficult to calculate. While they provide one possible approach at the end of their original paper, the elegance and simplicity of the non-graded approach is lost to the additional detail.\par

C\&F in its non-graded form therefore has middling applicability in the real world; however, it has served as the basis of much further work\cite{Herzig2009}, including more powerful modal logics.\cite{Kramdi}\par

It is clear to see that, between even these two early models of only one trait, a significant degree of difference exists in the implementation detail of the two formalisms. Moreover, their philosophies with regards social sciences differ; Marsh taking an approach between sociology and psychology, C\&F leaning more to the sociological aspects. However, both models are created with the intention of directing the behaviour of intelligent agents. Therefore, at least some similarities exist which might be used to construct methodologies and guidelines around.\par

\subsubsection*{Eigentrust}\label{sec:eigentrust}


\subsection*{Reputation}

\subsection*{Comfort}

\subsection*{Responsibility}


\section{Methodology}
\label{sec:methodology}

% How are we going to put the tools together?

% Identify commonalities between the models and the traits
% - Analysis of the models
% - Analysis of the traits being modelled

% Produce guidelines for model creation based on how thee models are made / have been made / can & should be made

% Produce design patterns for research and models based on these models

% Produce tooling and libraries which support these design patterns

% Produce source code for existing models created against these guidelines using these tools

% Produce new models which are suitable for industry-setting deployment
% - either smart cities or voice assistants

\section{Risks}
\label{sec:risks}
% The work's complex -- one set of guidelines might not be enough

% Creating these tools doesn't mean people will adopt them -- we'd need to get the word out, and that can cost! We'd need money for conferences and things. 

% 


\section{Project Management}
\label{sec:project_management}
% GTD-style management in a hierarchical structure


\section{Impact}
\label{sec:proposed_approach}
% Impact in two ways!


\subsection*{National Importance}
\label{sec:national_importance}
% Relates to EPSRC's push in ICT themes
% Improvements to British technological advancements
% Improvements in british urban development, and consumer electronics

\subsection*{Academic Impact}
\label{sec:academic_impact}
% Ties well into EPSRC's intention to grow funding opportunities in software engineering!
% Improvements to CS research, and HCI, and software engineering
% Improvements to many other fields, such as psych, socio, geography, philosophy, ethnography...



\let\oldbibliography\thebibliography
\renewcommand{\thebibliography}[1]{\oldbibliography{#1}
\setlength{\itemsep}{-3pt}}

\bibliographystyle{abbrv}
%\setstretch{0.8}
{
\scriptsize
\bibliography{biblio}
}
\end{document}
