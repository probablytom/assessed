\documentclass[a4paper,portrait,12pt,draft]{article}
\usepackage[latin1]{inputenc}
\usepackage{calc}
\usepackage{setspace}
\usepackage{fixltx2e}
\usepackage{graphicx}
\usepackage{multicol}
\usepackage[round]{natbib}
\usepackage{csquotes}
\usepackage{color}
\usepackage{hyperref}
\usepackage[colorinlistoftodos,textsize=tiny,obeyDraft]{todonotes}
\usepackage{cleveref}
 
\author{William Wallis --- 2025138W}
\title{Reviewing the Systematic Review}

\setlength{\oddsidemargin}{0.9847in-1in}
\setlength{\textwidth}{\paperwidth{}-0.9847in-0.9847in}

\crefname{chapter}{\S}{\S\S}
\crefname{section}{\S}{\S\S}
\setcounter{secnumdepth}{3}
\crefname{table}{table}{tables}
\Crefname{table}{Table}{Tables}
\crefname{figure}{figure}{figures}
\Crefname{figure}{Figure}{Figures}

\begin{document}
\maketitle

\begin{abstract}
Systematic reviewing is a technique for bringing scientific rigour to a computer science literature review, pioneered by Barbara Kitchenham~\citep{Kitchenham2004}. Specifically, Kitchenham's systematic reviews utilise concepts from the field of medical research to create literature reviews which are repeatable, and produce statistical and empirical results. The technique is posited as a tool for software engineering research. 12 years after Kitchenham's original guidelines were set for structuring a systematic literature review, the technique has seen widespread adoption --- but the original guidelines raise questions and note possible issues with the method. With a wide set of samples to choose from, a review of these systematic reviews may highlight whether these concerns are worth revisiting, before Kitchenham's guidelines --- or other methods derived from them --- become standard practice for the software engineering research community.
\end{abstract}

\section{Introduction}

\bibliographystyle{plainnat}
\bibliography{biblio}

\end{document}
