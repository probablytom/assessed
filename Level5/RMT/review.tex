\documentclass[a4paper,portrait,12pt,draft]{article}
\usepackage[latin1]{inputenc}
\usepackage{calc}
\usepackage{setspace}
\usepackage{fixltx2e}
\usepackage{graphicx}
\usepackage{multicol}
\usepackage[round]{natbib}
\usepackage{csquotes}
\usepackage{color}
\usepackage{hyperref}
\usepackage[colorinlistoftodos,textsize=tiny,obeyDraft]{todonotes}
\usepackage{cleveref}
 
\author{William Wallis --- 2025138W}
\title{Reviewing the Systematic Review}

\setlength{\oddsidemargin}{0.9847in-1in}
\setlength{\textwidth}{\paperwidth{}-0.9847in-0.9847in}

\crefname{chapter}{\S}{\S\S}
\crefname{section}{\S}{\S\S}
\setcounter{secnumdepth}{3}
\crefname{table}{table}{tables}
\Crefname{table}{Table}{Tables}
\crefname{figure}{figure}{figures}
\Crefname{figure}{Figure}{Figures}

\begin{document}
\maketitle

\begin{abstract}
Systematic reviewing is a technique for bringing scientific rigour to a computer science literature review, pioneered by Barbara Kitchenham~\citep{Kitchenham2004}. Specifically, Kitchenham's systematic reviews utilise concepts from the field of medical research to create literature reviews which are repeatable, and produce statistical and empirical results. 12 years after Kitchenham's original guidelines were set for structuring a systematic literature review, the technique has seen widespread adoption --- but the original guidelines raise questions and note possible issues with the method. With a wide set of samples to choose from, a review of these systematic reviews may highlight whether these concerns are worth revisiting, before Kitchenham's guidelines --- or other methods derived from them --- become standard practice for the software engineering research community.
\end{abstract}

\section{Introduction}
% Explore what a systematic review is, and why we're exploring them as a topic.
A Systematic Review is a literature review which collates the results of many papers, using statistical analysis to draw empirical results about the state of a research field and to answer research questions posited as the motivation for the review. Systematic reviews are born from the philosophy that a literature review should have scientific merit, and produce reproducible results, rather than standard techniques for literature reviews, which leave more room for subjective insight. The scientific nature of a systematic literature review, in theory, removes ambiguity and bias from literature review practice and lends the review the same validity and credence as a research study.\par

% Introduce the notion that there's a problem with the technique --- bring up the original guidelines.
Kitchenham's systematic literature review technique has begun to dominate as a literature review technique for software engineering. Kitchenham's review procedure stems largely from literature review techniques in medical research~\citep{Kitchenham2004, khan2001undertaking}, where empirical studies which verify the validity of literature already published in peer-reviewed journals is paramount to civilian safety. While the technique has seen widespread adoption, some issues exist with the implementation --- as noted by Kitchenham herself in a systematic review of systematic review procedures~\citep{Kitchenham2013}. More fundamentally, in Kitchenham's original guidelines there exist some notes which cast doubt on the suitability of a systematic review in the field of software engineering. For example:
\begin{displayquote}
    In particular, software engineering research has relatively little empirical research compared with the large quantities of research available on medical issues, and research methods used by software engineers are not as rigorous as those used by medical researchers.
\end{displayquote}
In her guidelines, Kitchenham provides types of empirical data which software engineering research \emph{does} produce which can be appropriate for analysis in a literature review. However, whether research rigour and types of data collected are make appropriate note of by the research community is clear only now that a wealth of systematic literature reviews have been produced.\par

% Whatcha doin?
In this review, a series of systematic literature reviews will be analysed and searched for their scrutiny of research rigour and format of empirical data. In this way, the importance of this doubt regarding the suitability of systematic reviews for software engineering research will be assessed.\par

The reviews chosen were picked as a result of their popularity on the ``\emph{Google Scholar}'' academic search engine, found by a search for ``software engineering ``systematic'' literature review'', and similar searches. This was to find papers which were well-cited and high-impact, because as the question to be answered would impact the culture around systematic reviews, these papers are important, as they are most likely to influence future systematic reviews.\par


\section{Effect Size Systematic Review}
\subsection{Systematic method}
First reviewed is a study into effect size in software engineering experiments~\citep{Kampenes2007}. The review was conducted according to systematic review guidelines --- the guidelines were not specified, however the procedure roughly aligned with that of Kitchenham, and Kitchenham's guidelines were cited as reasoning when defining a search criteria.\par

As required when developing a systematic review, \citep{Kampenes2007} used search criteria to determine the literature which was to be reviewed. This search criteria was cited from another paper (\citep{Sjoberg2005}, which shares some authors), rather than stated directly. \par

The method for data extraction was also well reported, as a systematic review should entail by Kitchenham's guidelines. The paper goes on to perform a deep and thorough statistical analysis, and concludes with a review of the results of these analyses with a comparison to the results of similar papers in Psychology and Behavioural Science. The paper succeeded in selecting several papers with empirical data so as to perform the statistical analysis with a large sample.\par

\subsection{Data extracted}
This paper confirmed Kitchenham's doubting note in this case. While a large sample was indeed selected after a time --- 92 papers over the course of 10 years --- selecting those papers required reviewing the contents of 5453 software engineering papers for suitable results. In the comparison with other fields, the similar Behavioural Science paper found 475 papers with suitable data to review. A similar education literature review found 226, published within a span of a single year.\par

Of those 5453 experiments --- 1.4\% of the original sample --- only 78 articles actually contained the controlled experiments sought by the authors. When compared to the similar work in other fields, computing science papers were thrice as likely to report effect size as education papers. However, education papers with suitable empirical experiments were published almost thirty times more frequently. Controlled experiments are therefore significantly less readily available than in other subjects where systematic reviews are a suitable method of literature review.\par

This does not dictate that systematic reviews should not be carried out in software engineering --- it does suggest, though, that empirical data might not be very generally available. It would therefore require researchers to wait a large span of time for suitable quantities of data to be produced to create a systematic review.\par


\section{Paper 2}



\section{Paper 3}



\section{Paper 4}



\section{Paper 5}



\section{Paper 6}



\section{Conclusion}

% Did the literature reviews matter?

% Things to note when writing this:
% - Do the literature reviews having ``systematic'' pedigree make them more susceptible to draw conclusions which aren't actually important?

\bibliographystyle{plainnat}
\bibliography{biblio}

\end{document}
