\documentclass[11pt,english,twocolumn,draft]{article}
\renewcommand{\familydefault}{\sfdefault}
\usepackage[T1]{fontenc}
\usepackage[utf8]{inputenc}
\usepackage{pslatex}
\usepackage[english]{babel}
\usepackage{blindtext}
\usepackage{setspace}
\usepackage{url}
%Definitions from Simon's mya4.sty
% Set the paper size to A4
\setlength{\paperheight}{297mm}
\setlength{\paperwidth}{210mm}
% Define commands which allow the width and height of the text
% to be specified. Centre the text on the page.
\newcommand{\settextwidth}[1]{
\setlength{\textwidth}{#1}
\setlength{\oddsidemargin}{\paperwidth}
\addtolength{\oddsidemargin}{-\textwidth}
\setlength{\oddsidemargin}{0.5\oddsidemargin}
\addtolength{\oddsidemargin}{-1in}
}
\newcommand{\settextheight}[1]{
\setlength{\textheight}{#1}
\setlength{\headheight}{0mm}
\setlength{\headsep}{0mm}
\setlength{\topmargin}{\paperheight}
\addtolength{\topmargin}{-\textheight}
\setlength{\topmargin}{0.5\topmargin}
\addtolength{\topmargin}{-1in}
}
\addtolength{\topsep}{-3mm}% space between first item and preceding paragraph.
\addtolength{\partopsep}{-3mm}% extra space added to \topsep when environment starts a new paragraph.
\addtolength{\itemsep}{-5mm}% space between successive items.

%End of Simon's mya4.sty
\usepackage{graphicx}%This is necessary and it must go after mya4
\settextwidth{176mm}
\settextheight{257mm}
\usepackage[usenames,dvipsnames,svgnames,table]{xcolor}
\def\baselinestretch{0.95}

\usepackage[compact]{titlesec}
\titlespacing{\section}{0pt}{*1}{*1}
\titlespacing{\subsection}{0pt}{*1}{*0}
\titlespacing{\subsubsection}{0pt}{*0}{*0}
\titlespacing{\paragraph}{0pt}{*0}{*1}
\titleformat*{\paragraph}{\itshape}{}{}{}
\usepackage[obeyDraft,textsize=tiny]{todonotes}
\usepackage{cleveref}

\begin{document}
\maketitle

\begin{abstract}
Systematic reviewing is a technique for bringing scientific rigour to a computer science literature review, pioneered by Barbara Kitchenham~\citep{Kitchenham2004}. Specifically, Kitchenham's systematic reviews utilise concepts from the field of medical research to create literature reviews which are repeatable, and produce statistical and empirical results. The technique is posited as a tool for software engineering research. 12 years after Kitchenham's original guidelines were set for structuring a systematic literature review, the technique has seen widespread adoption --- but the original guidelines raise questions and note possible issues with the method. With a wide set of samples to choose from, a review of these systematic reviews may highlight whether these concerns are worth revisiting, before Kitchenham's guidelines --- or other methods derived from them --- become standard practice for the software engineering research community.
\end{abstract}

\section{Introduction}

\bibliographystyle{plainnat}
\bibliography{biblio}

\end{document}
