\documentclass[a4paper,portrait,12pt,draft]{article}
\usepackage[latin1]{inputenc}
\usepackage{calc}
\usepackage{setspace}
\usepackage{fixltx2e}
\usepackage{graphicx}
\usepackage{multicol}
\usepackage[round]{natbib}
\usepackage{csquotes}
\usepackage{color}
\usepackage{hyperref}
\usepackage[colorinlistoftodos,textsize=tiny,obeyDraft]{todonotes}
\usepackage{cleveref}
 
\author{William Wallis --- 2025138W}
\title{Reviewing the Systematic Review}

\setlength{\oddsidemargin}{0.9847in-1in}
\setlength{\textwidth}{\paperwidth{}-0.9847in-0.9847in}

\crefname{chapter}{\S}{\S\S}
\crefname{section}{\S}{\S\S}
\setcounter{secnumdepth}{3}
\crefname{table}{table}{tables}
\Crefname{table}{Table}{Tables}
\crefname{figure}{figure}{figures}
\Crefname{figure}{Figure}{Figures}

\begin{document}
\maketitle

\begin{abstract}
Systematic reviewing is a technique for bringing scientific rigour to a computer science literature review, pioneered by Barbara Kitchenham~\citep{Kitchenham2004}. 12 years after Kitchenham's original guidelines were set for structuring a systematic literature review, the technique has seen widespread adoption --- but the original guidelines raise questions and note possible issues with the method. A review of these systematic reviews may highlight whether these concerns are worth revisiting, before Kitchenham's guidelines and those like them become standard practice for the software engineering research community.
\end{abstract}

\section{Introduction}
% Explore what a systematic review is, and why we're exploring them as a topic.
A Systematic Review is a literature review which collates the results of many papers, using statistical analysis to draw empirical results about the state of a research field and to answer research questions posited as the motivation for the review. Systematic reviews are born from the philosophy that a literature review should have scientific merit, and produce reproducible results. This technique somewhat opposes standard techniques for literature reviews, which leave more room for subjective insight. The scientific nature of a systematic literature review, in theory, removes ambiguity and bias from literature review practice and lends the review the same validity and credence as a research study, though a statistical analysis of collated results.\par

% Introduce the notion that there's a problem with the technique --- bring up the original guidelines.
Kitchenham's systematic literature review technique\footnote{Kitchenham's guidelines have undergone some revisions over time --- the two most cited versions in the papers reviewed being~\cite{Kitchenham2004} and~\cite{Kitchenham2007}. As the latter is an incremental improvement over the former, all of Kitchenham's guideline versions will be referred to through this document simply as Kitchenham's guidelines.} has begun to dominate as a literature review technique for software engineering. Conventional computing science literature reviews might closer resemble Webster's guidelines~\citep{Webster2002}, which are less rigorous, and less focused on empiricism and repeatability, yet offer structure to the review. However, doubts about systematic reviews are sometimes raised. For example, in Kitchenham's own guidelines:

\begin{displayquote}
    In particular, software engineering research has relatively little empirical research compared with the large quantities of research available on medical issues, and research methods used by software engineers are not as rigorous as those used by medical researchers.
\end{displayquote}~\cite{Kitchenham2004}

The quality of this data in real-world systematic reviews can only be apparent once systematic reviewing has become mainstream, which it now has. Therefore, in this review, a series of systematic literature reviews will be analysed and searched for their scrutiny of research rigour and format of empirical data. In this way, the importance of this doubt regarding the suitability of systematic reviews for software engineering research will be assessed. We will see that there is some reason for this doubt, and offer solutions to the problem as it manifests.\par

The reviews chosen were picked as a result of their popularity on the ``\emph{Google Scholar}'' academic search engine, found by a search for ``software engineering ``systematic'' literature review'', and similar searches. This was to find papers which were well-cited and high-impact, because as the question to be answered would impact the culture around systematic reviews, these papers are important, as they are most likely to influence future systematic reviews.\par


\section{Papers reviewed}
\subsection{Effect Size in Software Engineering}
One review to note is~\citet*{Kampenes2007} --- a review indicating it follows Kitchenham's guidelines and citing the paper, though never stating it outright. The search criteria used (which is necessary for a systematic review to note) was cited from another paper however (\cite{Sjoberg2005}, which shares some authors), rather than stated directly; the collated work was a set of software engineering papers exhibiting controlled experiments published over the span of 10 years.\par

The method for data extraction was well reported, as a systematic review should entail by Kitchenham's guidelines. The paper goes on to perform a deep and thorough statistical analysis, and concludes with a review of the results of these analyses with a comparison to the results of similar papers in Psychology and Behavioural Science. The paper succeeded in selecting several papers with empirical data so as to perform the statistical analysis with a large sample. The authors found 78 usable studies for their review, from the 5453 studies assessed. Happily, the researchers found quantitative data on which to perform some statistical analysis.\par

\subsection{Global Software Engineering}
% \subsubsection{Intent of Work}
\citet*{Smite2010} sets about the task of reviewing literature on global software engineering (GSE). Particularly, it attempts to collate and assess the results of literature which produce empirical data. The authors identify that there exists scarce literature on the topic, and so to collate the findings and categorise the growing yet important field, they employ a systematic review as a technique for categorising literature based on emerging trends.\par

As the field is young, this literature review serves to add to the literature present and to summarise the current state of the literature. It also contributes useful categories by which future research in the field might be defined --- as a literature review surveys much of the original research, this serves as a significant contribution to the field, directing future work.\par

The review guidelines used were from a Kitchenham-like standard \citep{Kitchenham2007}. The authors do not give a justification for the use of a systematic technique as opposed to a regular review. However, they do note that no systematic review yet existed --- so one is inclined to suppose that the authors sought to fill the niche they had identified. \citeauthor{Smite2010} do present a useful section explaining their search method, which they admit as broad.\par 

The authors found 387 papers which were possibly relevant, of which 59 were considered for their literature review.\par

% \subsubsection{Suitability of Systematic Approach}
% In the paper's abstract, the authors claim that:
% 
% \begin{displayquote}
%     ...the systematic review results in several descriptive classifications of the papers on empirical studies in GSE and also reports on some best practices identified from literature.
% \end{displayquote}~\citep{Smite2010}\par
% 
% This is true. However, the review also fails to produce useful data as due to ``\ldots{}the limited amount of data, statistical analysis was infeasible''. Due to the lack of statistical analysis, the results of the paper may as well have been produced by an ordinary review with specific search criteria. The systematic review process itself was useful in highlighting the need for research questions and search criteria, but the results born of the research did not produce the quantitative data a systematic review's value derives from. The same results could have come from a paper with no systematic requirement, but which borrowed a few techniques from systematic practice.\par
% 
% It is worth noting that, in this review, roughly 18\% of the literature found was suitable for the research at hand. These statistics fare much better than those of \citeauthor{Kampenes2007}. However, the original set of papers was less than 10\% of \citeauthor{Kampenes2007}'s set. Again in this case, systematic reviews eschew semantic insight for the repeatability of empirical study, but insufficient data exists to reliably and repeatedly carry out these review experiments.\par
% 

\subsection{Automated Analyses of Feature Models}
\citet*{Benavides} performs a review of techniques used for automating the analysis of a feature model. The review is an interesting blend of Kitchenham's systematic approach and the more semantic (yet structured) approach of Webster's guidelines. In doing this, \citeauthor{Benavides} produce a rigorous review which also persitently focuses on making its findings accessible to the reader, though data visualisation and detailing the non-quantitative findings of the review.\par

\citeauthor{Benavides} detail their effective research method as well as three research questions, which are closely examined. They also detail their inclusion and exclusion criteria for their search, enabling the repeatability of the systematic review. \par

The authors conclude their research, having assessed the state-of-the-art in the field and observing the current research trends, with an exposition as to future challenges the field may face. Unfortunately, the majority of their results are visualisation and discussion as opposed to statistical analysis, as per Kitchenham's method. These visualisations are however created in such a way that their review may be repeated and similarities observed, meaning that some degree of repeatability is still present.\par

\subsection{Motivation of Software Engineering} 
% Number of papers, Kitchenham method 
\citet*{Beecham2007} reviews how motivation affects the process of software engineering. Specifically, the study assesses how software engineers might me mtivated or unmotivated, and what factors cause this to happen on a personal level. It also assesses what properties software engineers tend to have, and what models of motivation exist in software engineering.\par

This study followed the Kitchenham guidelines for conducting their review. With five research questions answered over the course of the paper, the research is involved and lengthy --- however, no statistical analysis is produced. Instead, it uses representations of the collated research, alongside counts of traits of software engineers in the various studies. Kitchenham's guidelines are lauded in part for their rigour and repeatability. While this research contained both, the lack of data to analyse and semantic nature of the findings means that a repeatability study may not hold much value. Together with the many visual aids presented, this review may have benefited from techniques in Webster's method, which was was similar to in nature (yet systematic).\par

\subsection{Variability Management in Software Product Lines}
\citet*{Chen2007} review the field of Variability Management according to Kitchenham's guidelines. While there have been other literature reviews in the field, this systematic review is the first to rigorously assess it.\par

\citeauthor{Chen2007} select 34 papers from an original sample of 628. While the procedure to collect the original 628 is unclear, their inclusion and exclusion criteria from that point forward are well documented. In terms of adherence to systematic review guidelines, \citeauthor{Chen2007} do not succeed in producing a model or developing statistical inferences from the data produced. However, they do perform a textual analysis to create a repeatable assessment of the current state-of-the-art in Variability Management.\par

The paper also summarises problems with current literature: only a handful of publications studied tackled important issues such as scalability and testing. No mention of inspection and review as quality assurance techniques were observed by the authors. In this way, while no statistical analysis was completed, the authors successfully assess the current state of the literature and provide many points of potential improvement for researchers in the area.\par

\subsection{A Systematic Review of Systematic Reviews}
A final paper worth noting for the purposes of this review is Kitchenham's own systematic review of published systematic reviews \citep{Kitchenham2013}. The review by \citeauthor*{Kitchenham2013} is very thorough, and includes research questions, search criteria, and all other characteristic traits of a systematic review.\par

This review excluded papers to a pool of 45 from an original sample of 410. Like other reviewed papers, this review creates taxonomies and identifies key traits of a review through textual analysis rather than statistical analysis. However, the review also includes statistical analyses of these taxonomies, provided statistics such as Kappa calculations. It is the only paper reviewed to do so.\par

This paper selects papers with appropriate impacts and represents the state of the art in systematic reviews very effectively. In concluding the review, the authors highlight some concerns with current review procedure, as well a concerns with current guidelines and best practices, offering constructive criticism of changes to their own work and others in light of their findings. It also calls for further research into systematic reviews, and for better tooling to improve the issues they identify, which include ceasing to mention data extractors and checkers and to mention citation-based search strategies.\par



\section{Discussion}\label{sec:discussion}
% Worth pointing out that you can't subject a regular review to the same sort of scrutiny as a systematic review
% - At least you're guaranteed data and rigour as a reader
% - Maybe this is something semantic guidelines might be able to provide?

% Best thing we can say about all of these papers is that regardless of whether they use statistics, they're repeatable. 
% - Does repeatability matter in this case?
% - There may well be useful insight in the motivation case, for example, but simple counts of papers were presented instead. 
% - Results seem cursory this way --- never much food for thought, and a literature review is the best opportunity for thought food: someone's just read loads of papers and can tell us things we don't know about the field at large!

The systematic reviews selected have all been reviews of literature pertaining to a niche in software engineering. Interestingly, a common theme emerged, which was that there was frequently limited quantitative data for the reviewers to analyse statistically.\par

This seemed to occur for two reasons:
\begin{enumerate}
    \item Some subject areas did not easily produce empirical quantitative data. An example of this would be the~\cite{Benavides} article on Motivation; achieving consistently reliable quantitative data in psychological or ethnographic problem domains is difficult to do.
    \item The research culture of some subject areas did not seem inclined to promote the production and publication of quantitative data as a part of their work. Examples of this would be~\cite{Chen2007}, or~\cite{Smite2010}. 
\end{enumerate}

Unsurprisingly,~\cite{Kitchenham2013} succeeds in creating statistical analyses from the taxonomic data the authors collate over the course of the review. Also worth noting is that another paper,~\cite{Kampenes2007}, \emph{did} succeed in creating statistical measurements from the taxonomical data produced. One of the authors, however --- Dyba --- has their own systematic review procedure published~\citep{dybaguidelines}, so this comes at little surprise.\par

For systematic reviews to have a more scientific advantage over more orthodox review techniques, quantitative data evaluation is paramount. It remains the most reliable and clearest metric of a piece of scientific research's reproducibility.\par

For review culture where textual analysis is intended to go hand-in-hand with repeatability, a review method such as~\cite{Webster2002} may be more appropriate. Webster's method offers a helpful guide for getting away from the ``phonebook'' issues some literature reviews encounter. In addition, the repeatable nature of the taxonomical data they produce fits nicely in line with Webster's method.\par

An alternative method for solving these issues would be to adopt the statistical analysis methods from~\cite{Kitchenham2013} or~\cite{Chen2007}. This would allow at least modest statistical analysis of taxonomical data, should researchers be particularly inclined toward the systematic philosophy of literature reviewing.\par

Also worth noting is the high number of papers often needed to search to collect a sufficient sample to analyse from the search criteria Kitchenham advocates. Specifically:

\begin{table}[h]
\centering
\resizebox{\textwidth}{!}{%
\begin{tabular}{@{}lllllll@{}}
\toprule
                & \cite{Kampenes2007} & \cite{Smite2010} & \cite{Chen2007} & \cite{Benavides} & \cite{Kitchenham2013} & \cite{Beecham2007} \\ \midrule
Papers Found    &    5453      &  387                    &   628   &    72       &    410        &  519       \\
Papers Reviewed &     78     &    59                     &   34   &     53      &     45       &     92    \\
\% Utilised     &     1.43     &   15.24                  &  5.5    &   73.61        &  10.98          &   17.73      \\ \bottomrule
\end{tabular}%
}
\caption{Papers reviewed to achieve desired samples}
\end{table}
\section{Conclusion}

% Did the literature reviews matter?
The literature reviewed here, while of quality and all of impact, did not consistently fulfil the criteria which provides value to a systematic review. Indeed, Kitchenham's early doubt --- that software engineering might not produce enough quantitative data to bring scientific rigour to the review in this case --- seems well-placed in this instance.\par

Many solutions are available to remedy this, however. Guidelines more in line with statistical analysis of taxonomical data is one option --- of which~\cite{Kitchenham2013} and~\cite{Kampenes2007} are good examples. Other options would be the adoption of structured review methods where systematic reviews may be less appropriate, for which~\cite{Webster2002} would be a good choice. A final option would be, as a culture, to shift more toward quantitative data production and analysis\dots{} though this seems impractical as a solution.\par

% Things to note when writing this:
% - Do the literature reviews having ``systematic'' pedigree make them more susceptible to draw conclusions which aren't actually important? Can they get away with saying unimportant things, because the systematic review makes them sound more important?
% - Perhaps a systematic review should be something chosen only if a review turns out to produce lots of data to analyse? Should they necessarily start with rigour? If yes, how can we ensure more data of higher quality?


\bibliographystyle{plainnat}
\bibliography{biblio}

\end{document}
