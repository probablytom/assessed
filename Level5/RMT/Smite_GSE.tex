\subsection{Global Software Engineering}
% \subsubsection{Intent of Work}
\citet*{Smite2010} sets about the task of reviewing literature on Global Software Engineering (GSE). Particularly, it attempts to collate and assess the results of literature which produce empirical data. The authors identify that there exists scarce literature on the topic, and so to collate the findings and categorise the growing yet important field, they employ a systematic review as a technique for categorising literature based on emerging trends.\par

The review guidelines used were from a Kitchenham-like standard \citep{Kitchenham2007}. The authors do not give a justification for the use of a systematic technique as opposed to a regular review. However, they do note that no systematic review yet existed --- so one is inclined to suppose that the authors sought to fill the niche they had identified. \citeauthor{Smite2010} do present a useful section explaining their search method, useful for repetition of a systematic review. The authors found 387 papers which were possibly relevant, of which 59 were considered for their literature review.\par

As the field is young, this literature review serves to add to the literature present and to summarise the current state of the literature. It also contributes useful categories by which future research in the field might be defined --- as a literature review surveys much of the original research, this serves as a significant contribution to the field, directing future work. Overall, the work serves to help in guiding GSE research by collating existing research into the categories defined.\par


% \subsubsection{Suitability of Systematic Approach}
% In the paper's abstract, the authors claim that:
% 
% \begin{displayquote}
%     ...the systematic review results in several descriptive classifications of the papers on empirical studies in GSE and also reports on some best practices identified from literature.
% \end{displayquote}~\citep{Smite2010}\par
% 
% This is true. However, the review also fails to produce useful data as due to ``\ldots{}the limited amount of data, statistical analysis was infeasible''. Due to the lack of statistical analysis, the results of the paper may as well have been produced by an ordinary review with specific search criteria. The systematic review process itself was useful in highlighting the need for research questions and search criteria, but the results born of the research did not produce the quantitative data a systematic review's value derives from. The same results could have come from a paper with no systematic requirement, but which borrowed a few techniques from systematic practice.\par
% 
% It is worth noting that, in this review, roughly 18\% of the literature found was suitable for the research at hand. These statistics fare much better than those of \citeauthor{Kampenes2007}. However, the original set of papers was less than 10\% of \citeauthor{Kampenes2007}'s set. Again in this case, systematic reviews eschew semantic insight for the repeatability of empirical study, but insufficient data exists to reliably and repeatedly carry out these review experiments.\par
% 
